\chapter{Le Système D10}
%%%%%%%%%%%%%%%%%%%%%%%%%%%%%%%%%%%
%      FICHE DE PERSONNAGE        %
%%%%%%%%%%%%%%%%%%%%%%%%%%%%%%%%%%%
\section{Fiche de personnage}

% % % % % % % % % % % % % % % % % %
%  Créer une fiche de personnage  %
% % % % % % % % % % % % % % % % % %
\subsection{Comment créer une fiche de personnage}
La création de personnage est une des étapes les plus importantes du Jeu de Rôle, et le système D10 ne fait pas exception à la règle. 
Cependant, ce système a été créé à la fois pour être simple et rapide ; 
mais aussi pour permettre la représentation de tous les personnages que vous pourrez inventer.\\
De fait, la première étape dans la création de personnage est et reste l’invention de celui-ci. 
Rappelez-vous que le système de jeu n’est là que pour représenter ce que vous imaginez, et non pas pour dicter un personnage.\\

Une fois que vous aurez déterminé le personnage que vous souhaiterez incarner, c’est-à-dire son métier autant que son sexe, 
sa race, son physique, ses affinités, ses goûts, son apparence, et surtout son passé ; il convient évidemment de transformer cette idée 
-la plus précise possible- en une feuille de personnage. 
Pour cela, nul besoin de longues listes dans lesquelles chercher un sort qui ressemble au vôtre, 
il vous suffit de vous reporter dans les paragraphes suivants pour savoir comment répartir vos points afin de créer ce que vous avez imaginé.

% % % % % % % % % % % % % % % % % %
%            Les Races            %
% % % % % % % % % % % % % % % % % %
\subsection{Races}
Selon les Plans dans lesquels vous allez jouer, des races fantastiques peuvent ou pas exister. 
Dans l’hypothèse où ces races existent, aucune règle ne leur accorde de bonus ou malus raciaux. 
En revanche, selon la définition des races que l’on partage avec le Maître de Jeu, il convient de rester cohérent. 
Un Géant, tel qu’on le définit, est un être stupide d’environ cinq mètres de haut profitant de sa grande taille et de sa force naturelles 
pour acquérir le statut de prédateur. Il est évident qu’une telle créature ne peut raisonnablement avoir une agilité ou une intelligence 
égale à celle d’un elfe ou un Humain. S’il est possible qu’un de ces individus naisse plus intelligent que les autres, 
le milieu dans lequel il aura grandi s’il est resté parmi les siens est tel qu’il est plus qu’improbable qu’il ai développé cet aspect de sa personnalité.\\

C’est pourquoi, malgré qu’aucune règle ne l’indique, le choix de la race doit influer sur les caractéristiques des personnages, 
en cela que le passé du personnage ainsi que son physique sont directement influentes sur son intellect et ses aptitudes.

% % % % % % % % % % % % % % % % % %
%        Caractéristiques         %
% % % % % % % % % % % % % % % % % %
\subsection{Caractéristiques}
\subsubsection{Cas généraux}

Les Points de Caractéristiques d’un personnage sont à répartir entre huit caractéristiques. 
De manière générale, la moyenne humaine se situe entre deux et trois points. Un score négatif est impossible. 
Au-delà de cinq points, la capacité du personnage devient surnaturelle ; un nombre de points dans une caractéristique surpassant dix est inenvisageable.\\
Voici la liste des huit caractéristiques et leur descriptions :
\begin{itemize}
    \item[Force :] Il s’agit de la force physique du personnage autant que son endurance. Une force à zéro signifie une atrophie musculaire majeure. Le personnage est incapable de supporter jusqu’à son propre poids et nécessite une aide permanente. Une caractéristique à deux correspond à un humain plus faible que la moyenne, une force à cinq équivaut à un champion homme, de l’ordre du colosse tout de muscle capable de soulever plusieurs quintaux. Au-delà, la force du personnage ne peut plus être expliquée par une ascendance normale. Seules certaines races (Orcs, Trolls, Géants, Taurens, Loxodons…) peuvent justifier une force surnaturelle, ainsi que certaines formes de Magie (un nécromancien sous forme de squelette pourra posséder une force surnaturelle, par exemple).
    \item[Agilité :] Il s’agit de la dextérité du personnage, de ses réflexes ainsi que sa capacité d’esquive. En un mot, sa vitesse. Une agilité à zéro implique un personnage tellement lent qu’il lui est incapable de se mouvoir, une agilité de l’ordre de deux suppose un humain sans réflexes ou capacités particulières. Une agilité à cinq se traduit par un personnage terriblement véloce. Un être capable d’attraper une flèche au vol, de sauter plusieurs mètres et de courir à la vitesse d’un cheval au galop. Des exemples de races pouvant justifier une agilité surnaturelle seraient Viashno, Elfes, Kitsune…
    \item[Intelligence :] Intelligence d’un personnage résume à la fois sa Sagesse (ses connaissances) que sa capacité à apprendre. Une intelligence à zéro équivaut aux capacités cognitives du champ de patates ; à deux correspond un humain n’ayant jamais reçu d’éducation particulière. Une intelligence de l’ordre de cinq implique un érudit ayant passé sa vie à étudier, ou bien un génie capable de cogiter à une vitesse plus proche du supercalculateur que de l’humain normal. Au-delà […]
    \item[Perception :] La Perception d’un personnage exprime 
    \item[Volonté :] La Volonté d’un personnage représente, comme son nom l’indique, sa force de décision, mais aussi son entêtement et, par voie de conséquence, sa résistance à la suggestion, naturelle ou magique. Une Volonté à zéro correspond plus ou moins à la puissance d’esprit d’un Zombie. Une Volonté de l’ordre de deux équivaut à un humain crédule. […]
    \item[Psy :] Le Psy d’un personnage exprime son affinité avec les domaines magiques, quels qu’ils soient. 
    \item[Charisme :] Le Charisme d’un personnage résume à la fois sa prestance, son élocution, soit de manière générale sa propension à être apprécié, respecté, écouté. 
    \item[Chance :] La Chance d’un personnage représente son affinité avec les Dieux, et plus simplement la propension du personnage à éviter les désagréments aléatoires qui arrivent parfois dans une vie. […]
\end{itemize}
Comme vous pouvez le voir sur l’exemple joint de fiche de personnage; 
les caractéristiques sont présentées dans l’ordre suivant :
\begin{center}
\begin{tabular}{c|c}
    Force &Volonté \\
    Agilité & Psy \\
    \hline
    Intelligence & Charisme\\
    Perception & Chance
\end{tabular}
\end{center}
Cette répartition à pour avantage d’être divisible en quatre, comme indiqué. On observe alors quatre quarts très représentatifs de votre personnage. Ainsi dans le quart supérieur gauche (Force et Agilité) se résument les compétences Physiques. 
Dans le quart supérieur droit (Volonté et Psy) se situent les compétences Magiques. Dans le quart inférieur gauche (Intelligence et Perception) on observe les compétences Mentales, 
et enfin dans le quart inférieur droit se trouvent les compétences Sociales.

\subsubsection{Cas particuliers}
Certaines statistiques peuvent posséder un score de zéro mais toutefois refléter un autre état qu'une grave carence.
Ces statistiques doivent être achetées comme si elles étaient a un score de dix mais sont notés a zéro.
\begin{itemize}
    \item[Force :] un score de zéro dans ces conditions représente une absence d'enveloppe charnelle. Le personnage est présent sous la forme d'un spectre magique, reflet de sa volonté.
    \item[Psy :] un score de zéro représente un être totalement coupé de la magie, il reste toutefois notable qu'une boule de feu {\em d'essence magique} reste a l'appréciable température de plusieurs milliers de degrés, température connue notoirement pour sa capacité a tuer. 
    \item[Chance :] le personnage est littéralement, totalement et unilatéralement libre. Son destin ne peux être influencé par une éventuelle divinité. Son nom ne fût jamais inscrit dans le grand livre des {\em Chronarques}\footnote{\Cf{Les Chronarques}{Chronarques}}.
    \item[Volonté :] le personnage est totalement détaché du monde, nul chose ne peux l'attendre, nul désir n'a place en son cœur. Son esprit est au-delà du monde matériel.
\end{itemize}

% % % % % % % % % % % % % % % % % %
%   Points de Vie et de Fatigue   %
% % % % % % % % % % % % % % % % % %
\subsection{Points de Vie, Points de Fatigue}
\subsubsection{Fatigue}
Il s’agit de la résistance a l'épuisement, contrairement aux blessures, 
celle-ci dépend uniquement de la force, tout comme les blessures elle est repartie en 4 colonne (rien, essoufflé, éreinté, épuisement → coma). 
Chaque colonne possède autant que cases que la force de la personne.
Chaque coup inflige autant de blessures que de points de fatigue, si le coup est paré seulement le quart, 
arrondi à l'inférieur, et la moitié arrondie a l'inférieur si le coup est encaissé.

% % % % % % % % % % % % % % % % % %
%        Points de Folie          %
% % % % % % % % % % % % % % % % % %
\subsection{La Folie}
A chaque fois qu'une situation devient hors de contrôle ou d'appréhension pour un joueur 
il peut lui être demandé un jet de Volonté avec un degré de difficulté adapté à la situation, 
l'échec étant sanctionné par l'obtention d'un point de folie et/ou d'une folie passagère visant à fuir ou affronter sa terreur. 
Si le nombre de points de folie venait à dépasser la caractéristique Volonté (sa volonté innée) du personnage, 
celui-ci serait prit d'une folie destructrice et irréversible, qui amènerai à rendre le personnage injouable.

Il est important de noter que les points de folie viennent s'ajouter a la volonté innée du joueur pour calculer sa volonté totale, 
dans le contexte de tests de volonté (le joueur est aidé par sa folie pour résister a l'horreur).

% % % % % % % % % % % % % % % % % %
%          Compétences            %
% % % % % % % % % % % % % % % % % %
\subsection{Compétences}
Les compétences représentent les habiletés naturelles des personnages, 
elles vont de un à cinq d'ordinaire mais peuvent dépasser cinq pour des personnages maîtrisant une compétence au delà du naturel. 
Les compétences sont déterminées par le joueur et sont plus ou moins ciblées. 
Cela peut aller de Connaissance du monde jusqu'à la Connaissance d'une région particulière voir d'une ville ou d'une guilde. 
Les compétences regroupent l'ensemble du passé du joueur il ne faut donc jamais oublier les classique Orientation, 
Survie en foret et autre Combat a l'épée.\\
Toutes les compétences se payent au même prix, 
mais le Maître de Jeu pourra assigner une difficulté plus ou moins grande selon le la précision de la compétence.

\begin{quote}
Par exemple :\it Une compétence Maniement des armes à 5 points permettra à votre personnage d'être efficace avec toutes sortes d'armes, 
mais il sera en revanche moins performant qu'il ne l'aurait été avec une faux à deux mains et la compétence Maniement de la faux à deux mains, 
et ce même s'il n'avait eu que 2 points dans cette compétence.
\end{quote} 

\subsubsection{Compétences Expert et spécialisées}
Certaines compétences peuvent ou doivent être achetés au double du prix normal et seront définies comme des spécialités 
si elle sont rattachés a une autre compétence ; elle seront dites expertes si elles ne sont rattachées à aucune autre compétence. 
Dans le second cas elles seront rarement utilisées vu le degré de maîtrise, elle peuvent correspondre à une maîtrise innée ou à une excellence en la matière.

Les compétences spécialisées accordent quand à elles des degrés de maîtrise supplémentaires, en cas de réussite sur la compétence associée, 
il est possible d'utiliser la compétence de manière "passive". Pour la confirmer, il peut être nécessaire de faire le jet 
(Compétence Expert x2 + caractéristique)\footnote{Ce test devant rester assez rare puisque opposé au principe même de la caractéristique expert.}
 ou (Compétence spécialisée + Compétence associée + caractéristique).

% % % % % % % % % % % % % % % % % %
%            Artefacts            %
% % % % % % % % % % % % % % % % % %
\subsection{Artefacts}

%%%%%%%%%%%%%%%%%%%%%%%%%%%%%%%%%%%
%       NOTION DE CLASSES         %
%%%%%%%%%%%%%%%%%%%%%%%%%%%%%%%%%%%
\section{Notion de Classes}
\label{Classes}[Les différentes Classes]
\begin{enumerate}
    \item Insectes Misérables
    \item Pixies, Petites Créatures
    \item Taille humaine
    \item Géant, Drakôns, Trolls, Guivres
    \item Petits Dragons, Grands Drakôns, Géants des montagnes
    \item Dragons, Démons de 5\up{ème} cercle
    \item Grands Dragons, Divinités Mineures, Démons de 4\up{ème} cercle
    \item Anciens Dragons, Divinités, Démons de 3\up{ème} cercle
    \item Divinités Majeures, Démons de 2\up{nd} cercle
    \item Divinités Primordiales, Démons de 1\up{er} cercle
    \item[+] Elders Dragons, Très anciens dieux
\end{enumerate}

%%%%%%%%%%%%%%%%%%%%%%%%%%%%%%%%%%%
%       TEST DE COMPETENCE        %
%%%%%%%%%%%%%%%%%%%%%%%%%%%%%%%%%%%
\section{Test de compétence}
Il faut tout d'abord estimer la compétence et la caractéristique a associer a l'action puis d'estimer la difficulté à entreprendre cette même action par le ou les personnages:
\begin{center}
\begin{tabular}{llllllll}
Très Facile & Facile & Moyen & Difficile & Extréme & Ultime & Légendaire & Impossible\\
d6 & d8 & d10 & d20 & d30 & d100 & d1000 & d10000
\end{tabular}
\end{center}

L'étape suivante consiste a tirer le de et a regarder si celui-ci fait moins que la compétence plus la caractéristique associée. Le nombre de points sous le seuil représente les réussites.
\begin{center}
\begin{tabular}{ll}
0   & De justesse\\
1-2 & Passable\\
3-4 & Bon\\
5-6 & Très Bon\\
7-8 & Excellent\\
9+  & Exceptionnel
\end{tabular}
\end{center}
Un certain nombre de réussites peuvent êtres demandés pour certain tests. 
Les test opposés quand à eux comparent les degrés de réussite des deux parties, 
qui effectuent leurs tests avec les niveaux de difficulté relatifs à leurs propres conditions.
\begin{quote}
    Par exemple : Un joueur affronte un guerrier (non-joueur) au bras de fer  pour obtenir des renseignements. 
    Pour l'aider à vaincre, un des joueurs lance un sort d'entrave sur le guerrier, qui ne peut pas se défendre contre les sorts. 
    La difficulté pour le joueur est alors normale, tandis que pour le guerrier, qui souffre du sort d'entrave, le jet sera difficile. 
    En conséquence, le joueur utilise un D12, alors que son adversaire devra quand à lui utiliser un D20.
\end{quote}

%%%%%%%%%%%%%%%%%%%%%%%%%%%%%%%%%%%
%       REGLES DE COMBAT          %
%%%%%%%%%%%%%%%%%%%%%%%%%%%%%%%%%%%
\section{Règles de Combat}
% % % % % % % % % % % % % % % % % %
%       Combat a Distance         %
% % % % % % % % % % % % % % % % % %
\subsection{Combat à Distance}
% % % % % % % % % % % % % % % % % %
%    Combat au corps a corps      %
% % % % % % % % % % % % % % % % % %
\subsection{Combat au Corps à Corps}

%%%%%%%%%%%%%%%%%%%%%%%%%%%%%%%%%%%
%       REGLES DE MAGIE           %
%%%%%%%%%%%%%%%%%%%%%%%%%%%%%%%%%%%
\section{Règles de Magie}

% % % % % % % % % % % % % % % % % %
%         Magie innée             %
% % % % % % % % % % % % % % % % % %
\subsection{\em Magie innée}
La magie est généralement résolue comme un tir a distance, cependant il est bon de séparer les sort dans 4 catégories.

\subsubsection{Sorts d’entrave}
Selon le sort (passif ou actif) il est possible de tenter une esquive, le sort est lancé comme une attaque a distance avec (ou non) une réponse de la cible.

\subsubsection{Sorts d’agression}
Il s’agit de sorts ressemblant peu ou prou à une attaque de base, ce sort est résolu comme un duel normal, bien que l’assaillant ne puisse être blessé.

\subsubsection{Sorts de mort}
Ce genre de sort ne laisse que peu de chances à sa cible.
En général ces sorts possèdent un seuil élevé, de plus pour êtres lancés un total de points doit être accumulé. Une fois le sort lancé, 
un deuxième test (de précision) peut être demandé bien qu’en général la puissance du sort laisse ce genre de futilités un peu inutiles.
La nature du sort ne laisse que peu d’espoir quand aux chances de survie, aucune parade ou encaissement ne peut être sensément imaginée.

\subsubsection{Sort fixes}
Sort fixes, suspendu, glyphes ou autres enchantements.
Il s’agit en général de sorts d’entrave ou de mort (bien qu’un soin puisse être envisage), 
on conserve le résultat du lancé de sort de coté (lancé comme un sort de mort — soit une somme de d10), celui-ci représente la puissance du sort; et on tire séparément la marge (comme un test de compétence) communément nommée la transcendance ou chevauchement du sort, qui représente le maximum de puissance qui s'épuise au cours de son utilisation, cette marge est celui-ci est plutôt faible pour les sorts permanents et pseudo permanents (sorts rares et délicats a lettre en place), et est plutôt élevée pour les sorts limités dans le temps.

% % % % % % % % % % % % % % % % % %
%         Haute Magie             %
% % % % % % % % % % % % % % % % % %
\subsection{\em Haute Magie}
La Haute Magie est une magie caractérisée par trois Ars. Ars Locus, qui représente l'aire d'effet du sort, Ars Aetas qui réprésente la persistence de ce sort dans le temps ; et enfin Ars Virtus qui représente les dégats causés par ledit sort.\\
Le Haut Mage est défini par son affinité aux trois Ars, et par la même aux cinq couleurs de magie. En effet, les Ars représentent l'essence du sort et sont issus de la fusion des cinq couleurs de magie. Ainsi le Haut Mage qui base sa magie sur les trois Ars possède obligatoirement une affinité, plus ou moins grande, avec les cinq Couleurs de Magie.\\
\\
En terme de règles, un Haut Mage doit obligatoirement posséder sept compétences, dans lesquelles il est contraint d'investir {\em au moins} un point :
\begin{itemize}
\item[Ars Aetas]\Meta{ }
\item[Ars Locus]\Meta{ }
\item[Ars Virtus]\Meta{ }
\item[Magie Bleue]\Spe
\item[Magie Noire]\Spe
\item[Magie Rouge]\Spe
\item[Magie Verte]\Spe
\item[Magie Blanche]\Spe
\end{itemize}
Le Haut Mage n'est pas limité par une quelconque liste de sorts, ou même un domaine de compétences. Il a une totale liberté de création à chaque fois qu'il souhaite utiliser sa magie. En revanche, créer un sort de toutes pièces comme il le fait est loin d'être un processus aisé, et se divise en plusieurs étapes.\\
{\em Première étape :} Le Haut Mage détermine le sort qu'il souhaite lancer, et définit ainsi la couleur du sort et les rangs qu'il souhaite dans chaque Ars suivant la Table 6-B12. Le rang maximal que peut espérer un Haut Mage est directement déterminé par sa compétence dans cet Ars.\\
{\em Deuxième étape :} Le Haut Mage effectue un test de compétence dans chacun des Ars qu'il emploie, dont la difficulté est déterminée par le rang utilisé dans chaque Ars. L'Ars Aetas est lié à la caractéristique Intelligence, l'Ars Locus est lié à la Volonté et l'Ars Virtus au Psy.\\
Dans l'hypothèse d'un échec, ne fus-ce que dans l'un des trois Ars, entraîne l'échec du sort. Dans l'hypothèse d'une réussite, le nombre de degrés de réussite qu'il obtient est retenu séparément dans chacun des Ars et appellé Degré de Puissance. Ce degré est limité par l'affinité du mage avec la couleur, et cette limite va croissant avec le rang utilisé dans l'Ars. Par exemple, un Haut Mage ayant une affinité à 5 utilisant un Ars au rang 3 n'aura plus droit qu'à deux degrés de Puissance. Ce degré détermine la Puissance du sort selon chacun des Ars selon la table K-480.\\
\\
Afin d'augmenter ses chances de réussite, le Haut Mage peut choisir de se concentrer et de prolonger son incantation durant un nombre de tour limité par la valeur de sa Volonté. Chaque tour qu'il passe à incanter lui permet de réduire d'un degré la difficulté du test de compétence.
\\
\Ex{Considérons un excellent Haut Mage aux caractéristiques suivantes : Vol: 3 ; Int: 4 ; Psy: 5 ; Ars Virtus : 3 ; Ars Aetas : 1 ; Ars Locus : 2 ; Magie Rouge : 3 ; Magie Verte : 1 ; Magie Blanche : 1 ; Magie Bleue : 2 ; Magie Noire : 1.
\\Ce Haut Mage décide d'abord de lancer une boule de feu. Il s'agit donc d'une altération mineure (Ars Virtus au Rang 1), instantanée (Ars Aetas au rang 0) et de diamètre considérable (Ars Locus au rang 2). Etant donné qu'il s'agit d'une boule de feu, le sort relève du domaine de la magie rouge.\\
Le Haut Mage a donc deux jets à faire : Ars Virtus au d12 (d12 de base), Ars Locus au d20 (d12 de base + 1 difficulté).\\
Le Haut Mage décide d'incanter pendant toute la durée que lui permet sa volonté (3 tours) et réduit donc le sort de trois degrés de difficulté. Il choisit de réduire de deux degrés l'Ars Locus et d'un degré l'Ars Virtus. Il doit donc effectuer ses deux jets au d10.\\
{\em Ars Locus :} Le Haut Mage a une Volonté de 3, sa compétence d'Ars Aetas à 1 ; il doit donc faire moins de 4 avec un d10. D'autre part, son affinité avec la magie rouge étant de 3 et le rang de l'Ars de 2, son degré de réussite maximal est 1.\\
{\em Ars Virtus :} Le Haut Mage a un Psy de 5, et une compétence d'Ars Virtus a 3 ; il doit donc faire moins de 8 avec un d10. Il est limité par son affinité avec la magie rouge minorée par le rang de l'Ars, donc son degré de réussite maximal est 2.\\
Admettons qu'il obtient respectivement 1 puis 5 degrés de réussite à ses jets de dés. Le Haut Mage est donc limité dans son deuxième résultat et obtient 1 puis 2 réussites. Le mage lance donc une boule de feu de deux mètres de diamètre occasionnant 1D6 dommages.
}
\newcommand{\Amin}{$\mathbb{AM}$in }
\newcommand{\Amaj}{$\mathbb{AM}$aj }
\newcommand{\Afon}{$\mathbb{AF}$ }
\newcommand{\Caf}{$\wp$}
\newcommand{\Cam}{$\Im$}
\begin{center}
\begin{tabular}{|c| |l c| c | c |}
    \hline
    Ars & \multicolumn{2}{c}{Virtus} & {Locus} & Aetas\\
    \hline
    \hline
    0      & \Amin & (1 PV)         & 1 Cibles & 1 Rounds \\   
    1      & \Amin & (X PV)         & X Cibles & X Rounds \\
    2      & \Amin & (Xd6 PV) / (X PF)                                & Xd6 Cibles / \Cercle X m        & Xd6 Rounds \\
    3      & \Amin & (Xd6 PF) / (X \Caf)             & \Cercle 1d6m     &   X Minute\\
    4      & \Amaj & (Xd6 \Caf) / (X \Cam)  & \Cercle 1d6dm    &   X Heure \\
    5      & \Amaj & Xd6 \Cam    & \Cercle 1d6km    & X jour\\  
    6      & \Afon & - & - &  X mois\\
    7      & \Afon & - & - & X Mois \\
    8      & \Afon & - & - & X An \\
    \hline
    9     & \multicolumn{4}{c|}{Transcendance} \\
    \hline
\end{tabular}
\end{center}

\subsubsection{Temps d'incantation}
Il est possible d'incanter un sort un certain nombre de temps, dans le but d'en simplifier son lancer. 
La table suivante en explique les modalités. 

\begin{center}
\begin{tabular}{|c||l|l|}
    \hline
    Degrés & Temps (r) & Temps (s) \\
    \hline \hline
    1 & 1 & 6s \\ 
    2 & 3 & 18s \\ 
    3 & 7 & 42s \\
    4 & 15 & 1m 30s \\
    5 & 31 & 3m 6s \\
    6 & 95 & 1h 3m 30s \\
    7 & 351 & 5h 5m 6s \\
    8 & 1375 & 22h 5m 30s \\
    9 & 5471 & 3j 19h 1m 6s \\
    10 & 21855 & 15j 19h 1m 30s \\
    \hline
\end{tabular}
\end{center}

\subsubsection{Les sorts}
Les sorts sont des arcanes a travers les ars tellement pratiqués quelles en deviennent une évidence pour leur pratiquant, en obéissent donc a des règles légèrement différentes.
\paragraph{Créer un sort}
Il faut avoir au préalable déjà lancé le sort. Le nombre de degré en est son cout.
\paragraph{Lancer des sorts}
Chaque tour le mage accumule un nombre de points égal a deux fois sa caractéristique Psy. 
Un sort nécessite la moitié de son coût (d’achat) en points d’accumulation. 
Le mage ne peux lancer qu’un nombre de sorts égal a son intelligence.

\paragraph{Répondre a un sort}
A chaque fois qu’un mage lance un sort ce dernier va sur la Pile c’est a dire qu’il n’est pas résolu tout de suite, 
n'importe quel autre mage peut lancer un sort en réponse, par ordre d’initiative.
Important, une fois qu’on été lancés un nombre de sorts égal a votre intelligence, vous ne pouvez plus agir dans l'échange magique.
Les sorts se résoudrons dans l’ordre inverse a celui du lancement.



% % % % % % % % % % % % % % % % % %
%       Magie des Mots            %
% % % % % % % % % % % % % % % % % %
\subsection{\em Magie Dragonique}
La magie des mots est une matérialisation de la volonté du mage, celle ci transite par une composante vocale.\\
La caractéristique utilisée est uniquement l'intelligence, sous la forme d’un jet de compétence. 
Parler étant une Action libre, le mage peut prononcer un nombre de phrases égal a son intelligence et effectuer la moitie de son agilité en actions diverses.
\subsubsection{Choisir ses mots}
Chaque personne voit la magie d’une manière différente et chaque personne appréhende celle-ci par des manières différente.
Les mots que vous achetez dans votre voie sont a payer au coût normal, 
ceux qui vous sont proche sont a payer au coût expert et ceux opposée sont a payer au coût méta.
\subsubsection{Utiliser la magie des mots}
Pour lancer un sort a travers la magie des mots un jet de compétence doit être lancé avec comme caractéristique 
l’intelligence du lanceur plus la valeur du mot le plus faible.\\
Le nombre de degrés de réussite du résultat sera dépensé dans la table suivante.
\begin{center}
\begin{tabular}{|l| |c|c|c|}
    \hline
    Degré de réussite &     Virtus & Locus & Aetas\\
    \hline
    \hline
    0                 & 1 Pv       & 1 Cible & Aucune \\
    1                 & 1d6 PV     & 1d6 Cible & 1d6 Round \\
    2   & Altération (PF ou Carac physiques) & Cercle 1d6m R &   1 Minute\\
    4   & Altération Majeure (Carac Mentales)& Cercle 1d6dm R &   1 jour\\
    8   & Altération Fondamentale  &   Cercle 1d6km R & \\  
    16                &  -         & Illimité&  Infini dans le temps\\
    \hline
\end{tabular}
\end{center}

% % % % % % % % % % % % % % % % % %
%          Vivemagie              %
% % % % % % % % % % % % % % % % % %
\subsection{\em Vivemagie}

Porte principale (difficulté du dé), porte secondaire (nombre de réussites)
\begin{center}
\begin{tabular}{|l| |c|c||c| *{3}{|m{0.2\linewidth}}|}
\hline
Succès & \multicolumn{2}{c||}{Difficulté} & Adjectif & Pouvoir & Créativité & Volonté\\
\hline
\hline
2 & 0 & D10 & Moyen & Objet de l’ordre du mètre. & Objet grossier& N/A\\
4 & 1 & D12 & Difficile & Objet de deux ou trois mètres. & Objet assez ouvragé, début de précision. & Mouvement lent.\\
8 & 2 & D20 & Complexe & Dizaine de mètres, objet massif. & Précis, bien ouvragé. & Mouvement rapide. Grande concentration.\\
16 & 3 & D30 & Expert & Objet de l’ordre du tiers de cent-vingtaine de mètres. & Ouvrage de maître, extrèmement précis.Variation dans les matériaux. &
      Immence concentration, inusable, inaltérable, mouvement complexe.\\
32 & 4 & D100 & Maîtrise & Objet de l’ordre du dixième de millier de mètres & Ouvrage magnifique, chef-d’œuvres, matériaux complexes, travail d’horlogerie. &
      Mouvement autonome / conditionné, rapides et précis, prémices d’intelligence, immence durée de vie.\\
64 & 5 & D1000 & Impossible & Objet de l’ordre de la paire de milliers de mètres. & 
      Inimaginable, matériaux uniques, systèmes microscopiques, macro-systèmes complexes. & Indestructible, intelligence, durée de vie infinie.\\
\hline
\end{tabular} 
\end{center}
% % % % % % % % % % % % % % % % % %
%       Magie Tellurique          %
% % % % % % % % % % % % % % % % % %
\subsection{\em Magie Tellurique}

%%%%%%%%%%%%%%%%%%%%%%%%%%%%%%%%%%%
%        REGLES AVANCEES          %
%%%%%%%%%%%%%%%%%%%%%%%%%%%%%%%%%%%
\section{Régles avancées}
% % % % % % % % % % % % % % % % % %
%      Compétences Générales      %
% % % % % % % % % % % % % % % % % %
\subsection{Compétences Générales}
Perfection\Meta{ } - Permet de baisser la compétence d’un certain nombre de degrés de difficulté.
% % % % % % % % % % % % % % % % % %
%      Competences de Combat      %
% % % % % % % % % % % % % % % % % %
\subsection{Compétences de Combat}

\begin{itemize}
\item[\em Perfection] a
\item[\em Désarmement] a
\item[\em Esquive de combat] a
\end{itemize}

\subsubsection{Ecole du Tigre}
\begin{itemize}
\item[\em Coup désarmant] a
\item[\em Coup déchirant] a
\item[\em Combat synchronisé] a
\item[\em Frères d’armes] a
\item[\em Coup Handicapant] a
\item[\em Fureur du Tigre] a
\end{itemize}

\subsubsection{Ecole de l'Ours}
Discipline: Combat Brutal
\begin{itemize}
\item[\em Fureur de l'Ours] a
\item[\em Mâchoires de l'Alligator] a
\item[\em Coup Puissant]  a
\item[\em Encaissement] a
\item[\em Résistance] a
\item[\em Force surnaturelle] a
\item[\em Coup sonnant] a
\item[\em Brise Lame] a
\end{itemize}

\subsubsection{Ecole du Colibri}
Discipline: Combat acrobatique
\begin{itemize}
\item[\em Fureur de la Mante] a
\item[\em Ailes du Colibri] a
\item[\em Rafale de coups] a
\item[\em Coup double] a
\item[\em Rapidité surnaturelle]  a
\item[\em Coup déstabilisant] a
\item[\em Attaque tournoyante] a
\item[\em Combat multiple] a
\item[\em Riposte] a
\item[\em Fureur selon le Serpent] Vous pouvez infliger un certain nombre de vos Des sous forme de blessures plutôt que de les opposer a votre adversaire 
(les résistances s’appliquent normalement), les Des restants sont résolus normalement.
\item[\em Présence écrasante]\Meta{ } Perte de des de l’adversaire (même face)
\item[\em Coup Mortel] Si il devait advenir vous lanciez plus de Des que votre adversaire, 
vous pouvez infliger ces Des supplémentaires sous forme de blessures plutôt que de les opposer votre adversaire 
(les résistances s’appliquent normalement), les Des restants sont résolus normalement.
\item[\em Botte] a
\item[\em Coup de Grâce] a
\end{itemize}

\subsubsection{École du Dragon}
\begin{itemize}
\item[\em Prescience] a
\item[\em Fureur du Dragon] a
\item[\em Peau de la Salamandre] a
\item[\em Maîtrise du masque] a
\item[\em Bouclier mental] a
\item[\em Aspiration d’essence] a
\item[\em Esquive Magique] a
\item[\em Esprit de Tramepensée] a
\end{itemize}

% % % % % % % % % % % % % % % % % %
%     Competences Magiques        %
% % % % % % % % % % % % % % % % % %
\subsection{Compétences Magiques}
Y réfléchir a deux fois\Spe - Permet de lancer un jet de compétence avec un degré de difficulté supplémentaire, 
et de gagner un point de destin pour l’action si le test est réussi.
\subsubsection{Compétences lies a la magie des mots}
Élocution rapide\Spe - Permet de prononcer plus de phrases en un round
\subsubsection{Compétences lies a la Vivemagie}
\subsubsection{Compétences lies a la Haute magie}
Accumulation Magique\Spe - Permet d’accumuler la magie pendant un nombre de tour égal au niveau de la compétence + 1 sans prendre aucun risque.\\
Accumulation Magique\Meta{} - Cette accumulation est indétectable\\
Ars Locus, Ars Virtus, Ars Aetas - Compétentes limitant l’accès a certains hauts sorts.\\