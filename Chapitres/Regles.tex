\chapter{Le Système D10}
\section{Fiche de personnage}
\subsection{Comment créer une fiche de personnage}
La création de personnage est une des étapes les plus importantes du Jeu de Rôle, et le système D10 ne fait pas exception à la règle. Cependant, ce système a été créé à la fois pour être simple et rapide ; mais aussi pour permettre la représentation de tous les personnages que vous pourrez inventer.\\
De fait, la première étape dans la création de personnage est et reste l’invention de celui-ci. Rappelez-vous que le système de jeu n’est là que pour représenter ce que vous imaginez, et non pas pour dicter un personnage.\\
\\
Une fois que vous aurez déterminé le personnage que vous souhaiterez incarner, c’est-à-dire son métier autant que son sexe, sa race, son physique, ses affinités, ses goûts, son apparence, et surtout son passé ; il convient évidemment de transformer cette idée -le plus précise possible- en une feuille de personnage. Pour cela, nul besoin de longues listes dans lesquelles chercher un sort qui ressemble au vôtre, il vous suffit de vous reporter dans les paragraphes suivants pour savoir comment répartir vos points afin de créer ce que vous avez imaginé.
\subsection{Races}
Selon les Plans dans lesquels vous allez jouer, des races fantastiques peuvent ou pas exister. Dans l’hypothèse où ces races existent, aucune règle ne leur accorde de bonus ou malus raciaux. En revanche, selon la définition des races que l’on partage avec le Maître de Jeu, il convient de rester cohérent. Un Géant, tel qu’on le définit, est un être stupide d’environ cinq mètres de haut profitant de sa grande taille et de sa force naturelles pour acquérir le statut de prédateur. Il est évident qu’une telle créature ne peut raisonnablement avoir une agilité ou une intelligence égale à celle d’un elfe ou un Humain. S’il est possible qu’un de ces individus naisse plus intelligent que les autres, le milieu dans lequel il aura grandi s’il est resté parmi les siens est tel qu’il est plus qu’improbable qu’il ai développé cet aspect de sa personnalité.\\
C’est pourquoi, malgré qu’aucune règle ne l’indique, le choix de la race doit influer sur les caractéristiques des personnages, en cela que le passé du personnage ainsi que son physique sont directement influentes sur son intellect et ses aptitudes.
\subsection{Caractéristiques}
\subsubsection{Cas généraux}
Les Points de Caractéristiques d’un personnage sont à répartir entre huit caractéristiques. De manière générale, la moyenne humaine se situe entre deux et trois points. Un score négatif est impossible. Au-delà de cinq points, la capacité du personnage devient surnaturelle ; un nombre de points dans une caractéristique surpassant dix est inenvisageable.\\
Voici la liste des huit caractéristiques et leur descriptions :
\begin{itemize}
\item{Force :} Il s’agit de la force physique du personnage autant que son endurance. Une force à zéro signifie une atrophie musculaire majeure. Le personnage est incapable de supporter jusqu’à son propre poids et nécessite une aide permanente. Une caractéristique à deux correspond à un humain plus faible que la moyenne, une force à cinq équivaut à un champion homme, de l’ordre du colosse tout de muscle capable de soulever plusieurs quintaux. Au-delà, la force du personnage ne peut plus être expliquée par une ascendance normale. Seules certaines races (Orcs, Trolls, Géants, Taurens, Loxodons…) peuvent justifier une force surnaturelle, ainsi que certaines formes de Magie (un nécromancien sous forme de squelette pourra posséder une force surnaturelle, par exemple).
\item{Agilité :} Il s’agit de la dextérité du personnage, de ses réflexes ainsi que sa capacité d’esquive. En un mot, sa vitesse. Une agilité à zéro implique un personnage tellement lent qu’il lui est incapable de se mouvoir, une agilité de l’ordre de deux suppose un humain sans réflexes ou capacités particulières. Une agilité à cinq se traduit par un personnage terriblement véloce. Un être capable d’attraper une flèche au vol, de sauter plusieurs mètres et de courir à la vitesse d’un cheval au galop. Des exemples de races pouvant justifier une agilité surnaturelle seraient Viashno, Elfes, Kitsune…
\item{Intelligence :} L’Intelligence d’un personnage résume à la fois sa Sagesse (ses connaissances) que sa capacité à apprendre. Une intelligence à zéro équivaut aux capacités cognitives du champ de patates ; à deux correspond un humain n’ayant jamais reçu d’éducation particulière. Une intelligence de l’ordre de cinq implique un érudit ayant passé sa vie à étudier, ou bien un génie capable de cogiter à une vitesse plus proche du supercalculateur que de l’humain normal. Au-delà […]
\item{Perception :} La Perception d’un personnage exprime 
\item{Volonté :} La Volonté d’un personnage représente, comme son nom l’indique, sa force de décision, mais aussi son entêtement et, par voie de conséquence, sa résistance à la suggestion, naturelle ou magique. Une Volonté à zéro correspond plus ou moins à la puissance d’esprit d’un Zombie. Une Volonté de l’ordre de deux équivaut à un humain crédule. […]
\item{Psy :} Le Psy d’un personnage exprime son affinité avec les domaines magiques, quels qu’ils soient. 
\item{Charisme :} Le Charisme d’un personnage résume à la fois sa prestance, son élocution, soit de manière général sa propension à être apprécié, respecté… […]
\item{Chance :} La Chance d’un personnage représente son affinité avec les Dieux, et plus simplement la propension du personnage à éviter les désagréments aléatoires qui arrivent parfois dans une vie. […]\end{itemize}
\\
Comme vous pouvez le voir sur l’exemple joint de fiche de personnage ; les caractéristiques sont présentées dans l’ordre suivant :\\

TABLEAU

Cette répartition à pour avantage d’être divisible en quatre, comme indiqué. On observe alors quatre quarts très représentatifs de votre personnage. Ainsi dans le quart supérieur gauche (Force et Agilité) se résument les compétences Physiques. Dans le quart supérieur droit (Volonté et Psy) se situent les compétences Magiques. Dans le quart inférieur gauche (Intelligence et Perception) on observe les compétences Mentales, et enfin dans le quart inférieur droit se trouvent les compétences Sociales.
\subsection{Points de Vie, Points de Fatigue}
\subsection{Compétences}
\subsection{Artefacts}
\section{Notion de Classes}
\section{Test de compétences}
\section{Règles de Combat}
\subsection{Combat à Distance}
\subsection{Combat au Corps à Corps}
\section{Règles de Magie}
\subsection{Magie innée}
\subsection{Hautes Magie}
\subsection{Magie des Mots}
\subsection{Magie de Puissance}
\subsection{Dessin}