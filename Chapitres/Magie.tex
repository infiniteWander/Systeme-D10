\chapter{La Magie}
\section{De la Couleur de la Magie}
\label{Couleur}
\subsection{Le cercle des couleurs}
\subsection{La nature magique des joueur}
\section{Les cinq Magies}
\subsection{La {\em Magie innée}}
\subsection{La {\em Haute Magie}}
\subsection{La {\em Magie Dragonique}}
\subsection{La {\em Vivemagie}}
\subsection{La {\em Magie Tellurique}}
\section{Les Autres Magies}
\subsection{L'ancienne magie}
\subsection{Les Magies des Plans Extérieurs}
Les magies des plans extérieurs ne peuvent êtres pleinement utilisés sur ce plan.
Cependant la puissance des entités est en telle que celles-ci amènent une partie de leur magie avec elles.
\subsubsection{Invocation}
L'invocation a pour but d'amener totalement ou partiellement une entité des plans extérieurs, vers notre plan d'existence.

Une entité possède plusieurs Attributs (\Cf{Les couleurs de Magie}{Couleur}) ainsi qu'un rang divin (\Cf{Classe divine}{Classes}).\\
L'énergie que possédera l'invocation lors de son bref séjour sur ce plan est déterminé comme suit:\\
Le jet lors de l'invocation est ouvert et a la liberté de l'invocateur 
(il choisit lui même a quel tour l'arrêter), le seuil de difficulté est égal au rang de l'entité (il s’agit d’un jet de puissance).

L'entité arrive en jeu avec une énergie égale a l'affinité avec le joueur (a moduler avec les actions de ce dernier) $\times$ Score.
Ce qui représente score/rang minutes de présence, moins les actions effectués, qui coûtent respectivement:
\begin{itemize}
    \item 1 point: Actions banales
    \item 10 point: Actions normales
    \item 20 point: Actions incroyables
    \item 40 point: Actions impossibles
    \item 200 point: Actions demi-divines
\end{itemize}
